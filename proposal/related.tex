\section{Related Work}
Efficient FEM simulation plays an important role in designing deformable objects.
As mentioned, these problems are common in engineering and
graphics~\cite{Bendsoe2003,Bickel2010,Kou2012,Skouras2013,Chen2013,Xu2014}.
We can broadly partition the space of approaches for optimizing FEM simulation into two categories.
We term the first category \emph{numerical} approaches.
These methods use fast matrix inversion techniques and other insights about the algebra of the finite element method to increase its performance.
Simulators based on the multigrid method~\cite{Peraire1992,Zhu2010,McAdams2011} and Krylov subspace techniques~\cite{Patterson2012} have yielded impressive performance increases.
Other hierarchical numerical approaches, as well as highly parallel techniques, have also been applied to improve the time required to perform complex simulations~\cite{Farhat1991,Mandel1993}.
Finally, Bouaziz et al.~\cite{Bouaziz:2014} propose specially designed energy functions and an alternating time-integrator for efficient simulation of dynamics.

The second set of methods are \emph{reduction} approaches.
These algorithms attempt to intelligently decouple or remove degrees of freedom (DOFs) from the simulated system.
This leads to smaller systems resulting in a massive increase in performance, with some decrease in accuracy.
The proposed methods fall into this category.
Note, however, that numerical and reduction approaches need not be mutually exclusive.
For example, our algorithm may potentially be used as a preconditioner for a numerical approach. Algorithms based on reduction approaches mitigate the inevitable increase in error using one or more of three approaches: Adaptive remeshing, higher-order shape functions, or by adapting the constitutive model.

Adaptive remeshing alters the resolution of the simulation discretization in response to various metrics (stress, strain etc.).
Such methods seek to maintain an optimal number of elements and thus achieve reasonable performance.
Adaptive remeshing has proven useful for simulating thin sheets such as cloth~\cite{narain2012}, paper~\cite{narain2013}, as well as elastoplastic solids ~\cite{Wicke:2010} and solid-fluid mixtures~\cite{Clausen2013}.
More general basis refinement approaches have also been suggested~\cite{Debunne2001,Grinspun2002}.
While these methods do improve the performance of simulation algorithms, they have some drawbacks.
First, they often require complicated geometric operations which can be time consuming to implement.
Second, they introduce elements of varying size into the FEM discretization.
This can lead to poor numerical conditioning if not done carefully.
Finally, in order to maintain accuracy, it may still be necessary to introduce many fine elements, leading to slow performance.
Alternatively, one  can turn to P-Adaptivity for help.
This refers to adaptively introducing higher-order basis functions in order to increase accuracy during simulation~\cite{Szabo2004}.
Unfortunately, these methods suffer from requiring complicated mesh generation schemes and are not well-suited for iterative design problems.

An alternative approach to remeshing is to use higher order shape functions in order to more accurately represent the object's motion using a small set of DOFs.
Modal simulation techniques fall into this category~\cite{Shabana1991,Krysl2001,Barbic:subspace:2005}.
Substructuring ~\cite{Barbic2011} decomposes an input geometry into a collection of basis parts, performing modal reduction on each one. These basis parts can be reused to construct new global structures.
Other approaches involve computing physically meaningful shape functions as an offline preprocessing step.
For instance,  Nesme et al.~\cite{Nesme2009} compute shape functions based on the static configuration of a high resolution element mesh induced via a small deformation of each vertex.
Faure et al.~\cite{Gilles2011} use skinning transformations as shape functions to simulate complex objects using a small number of frames.
Gilles et al.~\cite{Faure2011} show how to compute material aware shape functions for these frame-based models, taking into account the linearized object compliance. Both Nesme et al.~\cite{Nesme2009} and Faure et al.~\cite{Faure2011} accurately capture material behavior in the linear regime, but, because their shape functions cannot change with the deformed state of the material, they do not accurately capture the full, non-linear behavior of an elastic object. Our non-linear metamaterials rectify this problem.
Computing material aware shape functions improves both the speed and accuracy of the simulation.
However, these methods require a precomputation step that assumes a fixed material distribution and geometry.
If the material distribution changes, these shape functions must be recomputed, and this becomes a bottleneck in applications that require constantly changing material parameters.  

The final coarsening technique involves reducing the degrees of freedom of a mesh while simultaneously augmenting the constitutive model at each element, rather than the shape functions.
Numerical coarsening is an extension of analytical homogenization which seeks to compute averaged material for heterogeneous structures ~\cite{guedes1990,farmer2002}.
Numerical coarsening, for instance, has been applied to linearly tetrahedral finite elements~\cite{Kharevych2009}.
These methods require an expensive precomputation step (a series of static solves) that must be repeated when the material content, or the geometry of an object changes.
%The deficits of such a simulation method, such as locking, are well documented~\cite{Ted2000,Muller2002}.
This holds these methods back from being suitable for iterative design problems. 

Recently, three methods have been introduced that are similar to ours in spirit.
Bickel et al.~\cite{Bickel2009} measured force-displacement to compute a spatially varying set of Young's moduli, interpolated in strain space.
Our work also involves learning new constitutive models for finite element methods with several key differences.
First, we present a more robust energy-based metamaterial model that does not require incremental loading during simulation.
Second, the previous work relies on captured data to build constitutive model, while we use a new sampling strategy that allows us to build our metamaterial model virtually.
This allows us to leverage large, high-performance compute clusters to speed up the process.
Finally, their work is completely geometry dependent---their computed material models cannot be transferred to new meshes.
Xu et al.~\cite{Xu2014} and Li et al.~\cite{Li2014} computed material distribution given user specified forces and displacements.
	They compute materials in a low-dimensional space of material modes to speedup and regularize the solution.
	However, their method requires a known geometry, and furthermore,
	users cannot control per-element material assignment.
	Rather than computing material distribution,
	our method supports user specified topology changes and material assignment.
	These features are important due to the design-centric nature of our work.

Data-driven techniques have also been applied to add fine detail to coarse surgical simulations \cite{Cotin1999,Kavan:2011,Seiler2012}. While these methods do not help with the type of iterative design problems addressed here, they could be combined with our fast coarsening in order to quickly ``upscale'' coarsened simulation results to original simulation resolution.