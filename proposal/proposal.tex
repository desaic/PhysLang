\documentclass[11pt]{article}
\usepackage[margin=1in]{geometry}
\title{ \large{Massachusetts Institute of Technology\\~\\
	Department of Electrical Engineering and Computer Science\\~\\~\\
	Proposal for Thesis Research in Partial Fulfillment\\~\\
	of the Requirements for the Degree of\\~\\
	Doctor of Philosophy }}
\date{}
\begin{document}
	\maketitle
\begin{flushleft}
 Title: Multiscale Methods for Simulation and Fabrication of Deformable Objects\\~\\
\end{flushleft} 
\hskip-0.2cm\begin{tabular}{p{4cm} p{6cm} p{8cm}}
 	 Submitted by: & Desai Chen & \underline{\hspace{6cm}}\\
                   &32 Vassar St., 32-311 & (Signature of author) 	  \\
                   &Cambridge, MA 02139&
\end{tabular}\\
\begin{flushleft}
Date of submission: May , 2016\\~\\
Expected date of completion: June 2016\\~\\
Laboratory where thesis will be done: MIT CSAIL\\~\\
Brief statement of the problem:\\~\\
Crafting the behavior of a deformable object is difficult---whether it is a biomechanically accurate character model or a multimaterial 3D print design.
Getting it right requires constant iteration, performed either manually or driven by an automated system.
Throughout this process the geometry and material composition of the object are in constant flux.
Accurate simulation of deformable objects using standard finite element methods is computationally expensive especially for large deformations beyond linear elasticity regime.
We propose to speed up simulation using multiscale methods that approximates the solution at coarser scales while capturing the overall behavior with sufficient accuracy for design tasks.
\end{flushleft}

\end{document}