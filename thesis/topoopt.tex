\chapter{Computation Design with Microstructures}
Many engineering problems focus on the design of complex structures that needs to meet high level objectives such as the capability to support localized stresses, optimal tradeoffs between compliance and mass, minimal deformation under thermal changes, etc. One very popular approach to design such structures is topology optimization.
Topology optimization generally refers to discretizing the object of interest into small elements and optimizing the material distribution over these elements in such a way that the functional goals are satisfied \cite{bendsoe2004topology}. Traditionally, topology optimization focused on designs made of homogeneous materials and was concerned with macroscopic changes in the object geometry.
With the advent of multi-material 3D printing techniques, it is now possible to play with materials at a much higher resolution, allowing to obtain much finer designs and, thus, improved functional performances.
Unfortunately, standard techniques for topology optimization do not scale well and they cannot be run on objects with billions of voxels. This is because the number of variables to optimize increases linearly with the number of cells in the object. Since many current 3D printers have a resolution of 600DPI or more, a one billion voxel design occupies only a 1.67 inch cube.

Most previous algorithms working in the material space focused on optimizing a single material property such as density or material stiffness, for which analytical formulas describing the property bounds exist \cite{Allaire93Bounds}. On the contrary, optimizing the structure and material distribution of an object in a high dimensional material property space remains an open problem. In this work, we propose a new computational framework for topology optimization with microstructures that supports design spaces of multiple dimensions. We start by computing the gamut of the material properties of the microstructures by alternating stochastic sampling and continuous optimization. This gives us a {\it discrete} representation of the set of achievable material properties, from which we can construct a {\it continuous} gamut representation using a level set field. We then reformulate the topology optimization problem in the continuous space of material properties and propose an efficient optimization scheme that finds the optimized distributions of multiple material properties simultaneously inside the gamut.
Finally, in order to obtain fabricable designs, we map the optimal material properties back to discrete microstructures from our database.

Our general formulation can be applied to a large variety of problems. We demonstrate its efficacy by designing and optimizing objects in different material spaces using isotropic, cubic and orthotropic materials. We apply our algorithm to various design problems dealing with diverse functional objectives such as minimal compliance and target strain distribution. Furthermore, our approach utilizes the high-resolution of current 3D printers by supporting designs with trillions of voxels. We fabricate several of our designs, thus, demonstrating the practicality of our approach.

The main contributions of our work can be summarized as follows:
\begin{itemize}
	\item We present a fully automatic method for computing the space of material properties achievable by microstructures made of a given set of base materials.
	\item We propose a generic and efficient topology optimization algorithm capable of handling objects with a trillion voxels. The key of our approach is a reformulation of the problem to work directly on continuous variables representing the material properties of microstructures. This allows us to cast topology optimization as a reasonably sized constrained optimization problem that can be efficiently solved with state of the art solvers.
	\item We validate our method on a set of test cases and demonstrate its versatility by applying it to various design problems of practical interest.
\end{itemize}
\section{Overview}
Given as input a set of base materials, an object layout, and functional objectives, the goal of our system is to compute the material distribution inside the object in order to optimize these functional objectives. In our approach, we do not solve the problem directly, instead we work with microstructures made of the base materials and the space of physical material properties spanned by them. The complete pipeline of our system, illustrated in Figure \ref{fig:overview}, can be decomposed into three stages.

\paragraph{Material Space Precomputation}
In the first stage, we estimate the gamut of material properties covered by all possible microstructures made by spatial arrangement of base materials. 
Since exhaustively computing the properties of all these microstructures is, in practice, intractable, we progressively increase the material space by alternating a stochastic search and a continuous optimization. The first step introduces discrete changes in the materials of the microstructures and allows emergence of new types of microstructures. The second step allows to locally push the material space boundaries by refining the microstructure shapes. After completing this stage, we obtain a discrete representation of the space of material properties and the mapping between these properties and the corresponding microstructures.

\paragraph{Gamut-based Continuous Topology Optimization}
In the second stage, we construct a smooth continuous gamut representation of the material property space by using a level set field. We define our topology optimization problem directly in this space. Our approach minimizes the objective function over possible material parameters while asking for strict satisfaction of the physics constraints -- typically, the static equilibrium -- as well as the strict satisfaction of the physical parameter bounds. Taking advantage of our gamut representation as a level set, we formulate this last constraint as limiting the material properties to stay on the negative side of the level set. This guarantees that the material properties that we use in the optimization are always physically realizable.

\paragraph{Fabrication-oriented Microstructure Mapping}
In the last stage, we generate a printable result by replacing each cell in the object layout with a microstructure whose material properties are the closest to the continuous material assignment resulting from the optimization. We also take into account the boundary similarity across adjacent cell interfaces to improve the connectivity between microstructures. This results in a complex, high-resolution, multi-material model with optimized functional specifications.