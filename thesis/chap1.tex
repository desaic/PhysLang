\chapter{Introduction}
Crafting the behavior of a deformable object is difficult---whether
it is a biomechanically accurate character model or a new multimaterial
3D printable design.
Getting it right requires constant iteration,
performed either manually or driven by an automated system.
These design iterations often require many expensive physical prototypes
that takes months to make and test.
Physics-driven computational design can be used to reduce the time and cost of designing such systems.

Instead of using physical prototypes, predictive physics-based simulation can be used to evaluate the performance of given designs.
The gold standard technique for estimating the mechanical behavior
of a deformable object under load is the finite element method
(FEM).
While accurate, FEM is notoriously slow, making it a major
bottleneck in the iterative design process.
This is because FEM is only accurate with a sufficiently high-resolution discretization.
In order to apply FEM to design problems, we require simulation techniques that are fast and accurate even in the face of constant tinkering.
In this thesis, we propose to achieve efficiency using low-resolution meshes while ensuring accuracy by computing new material models for the coarse finite elements.
\section{Background}
Computational design concerns itself with optimizing the shape and material assignment in an object in order to control its large scale behavior.
Advances in computational design, physical modeling and rapid manufacturing 
have enabled the automated design and fabrication of objects with customized physical properties.
The range of media and applications addressed in previous literature is very diverse spanning optical properties, inertia, strength, elasticity etc.
In the area of designing optical properties, \citet{Hasan:2010:PRO} and
\citet{Dong:2010:FSS} provided methods for printing objects with desired subsurface scattering properties. Objects have been made to reflect or bend light to create custom images \citep{papas11goal,Weyrich:2009,kiser2013}.
The inertia tensor of an object has been designed for stable standing~\citep{Prevost:2013kb}, floating~\citep{musialski-2015}
and spinning~\citep{Bacher:2014}.
The aerodynamic behavior of rigid mechanisms have been considered to design  paper airplanes~\citep{Umetani:2014}, kites~\citep{Martin:2015}, and multi-copters~\citep{Du:2016}.
For 3D-printed objects, many researchers have investigated how to improve structural strength~\citep{Stava:2012,Zhou:2013,Langlois:2016,Wu:2016,Ulu:2017}.
More complex mechanisms have been designed to achieve specified motions using kinematics~\citep{Zhu:2012,Coros:2013:CDM,Bacher:2015,Megaro:2017}.

The most relevant line of work is the design and fabrication of deformable objects with prescribed behavior under load.
The behavior is usually controlled by altering the material composition and shape of the designs.
\citet{Bickel:2010:DAF} used a measurement based material model to design layered soft structures with prescribed non-linear static deformation behavior.
\citet{Chen:2014:ANM} optimized the rest shape of a deformable object such that it deforms the right way under gravity.
Later, researchers use material composition and geometry to control characters 
articulated with kinematics~\citep{Bickel:2012,Skouras13Computational}.

Like these works, our system also allows to match given input deformations.
 However, while these previous systems assume a small set of available base 
 materials and use these base materials in relatively coarse discretizations, 
 our system combines the base materials into microstructures to expand the 
 design possibilities. Also relevant is the tool presented by Xu et al. [2015] 
 that allows to interactively design heterogeneous
materials for elastic objects subject to prescribed displacement and
forces. However, their method does not target fabricable objects
specifically and might produce materials that are not available in
the real world. In an effort to unify individual contributions when
dealing with inverse modeling problems, Chen et al. [2013]

\section{Motivations}


\section{Description}\label{ch1:desc}


\subsection{}


\subsection{}

\section{}


\subsection{}

\subsection{}

\section{}

\subsection{}

\subsection{}
