\chapter{Introduction}
Crafting the behavior of a deformable object is difficult---whether
it is a biomechanically accurate character model or a new multimaterial
3D printable design.
Getting it right requires constant iteration,
performed either manually or driven by an automated system.
These design iterations often require many expensive physical prototypes
that takes months to make and test.
Physics-driven computational design can be used to reduce the time and cost of designing such systems.

Instead of using physical prototypes, predictive physics-based simulation evaluates the performance of virtual designs to accelerate the design process.
The gold standard technique for estimating the mechanical behavior
of a deformable object under load is the finite element method
(FEM).
While accurate, FEM is notoriously slow, making it a major
bottleneck in the iterative design process.
This is because FEM is only accurate with a sufficiently high-resolution discretization.
In order to apply FEM to design problems, we require simulation techniques that are fast and accurate even in the face of constant tinkering.
In this thesis, we propose to achieve efficiency using low-resolution meshes while ensuring accuracy by computing new material models for the coarse finite elements.
\section{Computational Design}
Computational design concerns itself with optimizing the shape and material assignment in an object in order to control its large scale behavior.
Advances in computational design, physical modeling and rapid manufacturing 
have enabled the automated design and fabrication of objects with customized physical properties.
The range of media and applications addressed in previous literature is very diverse spanning optical properties, inertia, strength, elasticity etc.
In the area of designing optical properties, \citet{Hasan:2010:PRO} and
\citet{Dong:2010:FSS} provided methods for printing objects with desired subsurface scattering properties. Objects have been made to reflect or bend light to create custom images \citep{papas11goal,Weyrich:2009,kiser2013}.
The inertia tensor of an object has been designed for stable standing~\citep{Prevost:2013kb}, floating~\citep{musialski-2015}
and spinning~\citep{Bacher:2014}.
The aerodynamic behavior of rigid mechanisms have been considered to design  paper airplanes~\citep{Umetani:2014}, kites~\citep{Martin:2015}, and multi-copters~\citep{Du:2016}.
For 3D-printed objects, many researchers have investigated how to improve structural strength~\citep{Stava:2012,Zhou:2013,Langlois:2016,Wu:2016,Ulu:2017}.
More complex mechanisms have been designed to achieve specified motions using kinematics~\citep{Zhu:2012,Coros:2013:CDM,Bacher:2015,Megaro:2017}.

The most relevant line of work is the design and fabrication of deformable objects with prescribed behavior under load.
The behavior is usually controlled by altering the material
composition and shape of the designs.
\citet{Bickel:2010:DAF} used a measurement based material model to design 
layered soft structures with prescribed non-linear static deformation behavior.
\citet{Chen:2014:ANM} optimized the rest shape of a deformable object such that 
it deforms the right way under gravity.
Later, researchers use material composition and geometry to control 
articulated characters~\citep{Bickel:2012,Skouras13Computational}.
Like these works, one of our goals is to efficiently design compliant objects 
with desired deformation behavior. These work assume a small set of available 
base materials such a stiff and a soft material.
To expand the range of material building blocks, the base materials are 
organized into larger structures known as microstructures.
These microstructures are then used in design optimization loop to 
compose objects with specified compliant 
properties~\citep{Schumacher:2015,Panetta:2015,Zhu:2017:TTO}.

\section{Efficient FEM Simulation}
As mentioned, design and fabrication of deformable objects is common in
engineering and graphics~\citep{bendsoe2004topology,Kou2012,McAdams2011}.
Efficient FEM simulation plays an important role in automating the design process.
We can broadly partition the space of approaches for
optimizing FEM simulation into two categories. We term the first
category numerical approaches. These methods use fast matrix inversion
techniques and other insights about the algebra of the finite
element method to increase its performance. Simulators based on
the multigrid method~\citep{Peraire1992,Zhu2010,McAdams2011}
and Krylov subspace techniques~\citep{Patterson2012}
have yielded impressive performance increases. Other hierarchical
numerical approaches, as well as highly parallel techniques, have
also been applied to improve the time required to perform complex
simulations~\citep{Farhat1991,Mandel1993}. Finally, \citet{Bouaziz:2014} propose specially designed energy functions and an
alternating time-integrator for efficient simulation of dynamics.

The second set of methods are reduction approaches. These algorithms
attempt to intelligently decouple or remove degrees of
freedom (DOFs) from the simulated system. This leads to smaller
systems resulting in a massive increase in performance, with some
decrease in accuracy. Our algorithm falls into this category. Note,
however, that numerical and reduction approaches need not be mutually exclusive.
For example, since our algorithm uses the same type of spatial discretization as the underlying FEM,
faster numerical algorithms such as multigrid will improve the efficiency of our algorithm.
Algorithms based on reduction approaches mitigate the inevitable increase in error using
one or more of three approaches: Adaptive remeshing, higher-order
shape functions, or by adapting the constitutive model.

Adaptive remeshing alters the resolution of the simulation discretization
in response to various metrics (stress, strain etc.). Such methods
seek to maintain an optimal number of elements and thus achieve
reasonable performance. Adaptive remeshing has proven useful
for simulating thin sheets such as cloth~\citep{Narain2012}, paper~\citep{Narain2013}, as well as elastoplastic solids~\citep{Wicke:2010} and solid-fluid mixtures~\citep{Clausen2013}.
More general basis refinement approaches have also been suggested~\citep{Debunne2001,Grinspun2002}.
While these methods do improve
the performance of simulation algorithms, they have some drawbacks.
First, they often require complicated geometric operations
which can be time consuming to implement.
Second, they introduce elements of varying size into the FEM discretization.
This can lead to poor numerical conditioning if not done carefully.
Finally, in order to maintain accuracy, it may still be necessary to introduce many
fine elements, leading to slow performance.
Alternatively, one can turn to P-Adaptivity for help.
This refers to adaptively introducing higher-order basis functions 
in order to increase accuracy during simulation~\citep{Szabo2004}.
Unfortunately, these methods suffer from requiring complicated mesh generation schemes and are not
well-suited for iterative design problems.
An alternative approach to remeshing is to use higher order shape
functions in order to more accurately represent the object's motion
using a small set of DOFs. Modal simulation techniques fall into
this category~\citep{Shabana1991,Krysl2001,Barbic:subspace:2005}.
Substructuring~\citep{Barbic2011} decomposes an input
geometry into a collection of basis parts, performing modal reduction on each one.
These basis parts can be reused to construct new global structures.
Other approaches involve computing physically meaningful shape functions as an
offline preprocessing step.
For instance, \cite{Nesme2009} compute shape functions based on
the static configuration of a high resolution element mesh induced
via a small deformation of each vertex.
\cite{Faure2011} use skinning transformations as shape functions to simulate complex
objects using a small number of frames.
\cite{Gilles2011} show how to compute material aware shape functions for these framebased
models, taking into account the linearized object compliance.
Both \cite{Nesme2009} and \cite{Faure2011} accurately capture
material behavior in the linear regime, 
but, because their shape functions cannot change with the deformed state of the material,
they do not accurately capture the full, non-linear behavior of an
elastic object.
Our non-linear metamaterials rectify this problem.
Computing material aware shape functions improves both the speed
and accuracy of the simulation. However, these methods require a
precomputation step that assumes a fixed material distribution and geometry.
If the material distribution changes, these shape functions
must be recomputed, and this becomes a bottleneck in applications
that require constantly changing material parameters.
The final coarsening technique involves reducing the degrees of freedom
of a mesh while simultaneously augmenting the constitutive
model at each element, rather than the shape functions.
Numerical coarsening is an extension of analytical homogenization which seeks
to compute optimal,
averaged material for heterogeneous structures~\citep{Guedes1990,farmer2002}.
Numerical coarsening, for instance, has been applied to linearly
(in terms of material displacement)
elastic tetrahedral finite elements~\citep{Kharevych2009}.
These methods require an expensive precomputation step (a series
of static solves) that must be repeated when the material content, or
the geometry of an object changes.
This holds these methods back from being suitable for iterative design problems.
\section{Topology Optimization}
Topology optimization is concerned
with the search of the optimal distribution of one or more materials 
within a design layout while satisfying given constraints~\citep{bendsoe2004topology}.
These methods typically work with linear elastic material since it is easier to differentiate the objective function with respect to material distribution.
They are initially applied to structural design problems~\citep{bendsoe:1989:optimal} where the deformation is often small enough to be modeled by linear elasticity.
Topology optimization has been extended since then to a variety of problems including compliant mechanism design~\citep{Sigmund97Compliant},
mass transfer~\citep{challis:2009:level},
metamaterial design~\citep{Sigmund2000,cadman:2013:design},
multifunctional structure design~\citep{yan:2015:two},
and coupled structure-appearance optimization~\citep{Martinez:2015:SAO}.
Many algorithms have been proposed to numerically solve the optimization problem itself.
We refer to the book by~\citet{sigmund:2013:topology} for a complete review.
In the very popular SIMP (Solid Isotropic Materials with Penalization) method,
the presence of material in a given cell is controlled by locally varying its density.
A binary design is eventually achieved by penalizing intermediate values for these densities.
In practice, this method works well for two-material
designs (e.g., a material and a void), but generalizing this method
to robustly handle higher dimensional material spaces remains challenging.
Another class of methods rely on homogenization.
They virtually replace the material in each voxel of the object by a mixture of the base materials that is known to be locally optimal for the
original topology problem~\citep{Allaire2012}.
While very powerful, these methods are specialized for the minimum compliance problem, for
which the link between stiffness and optimal density can be analytically formulated.
In a sense, our work can been seen as a generalization of this type of approaches to handle arbitrary materials in broader contexts.

While broadly used in engineering, standard methods for topology optimization suffer from a major drawback.
The parametrization of the problem at the voxel level makes them extremely expensive and impedes their use on high resolutions models such as the ones generated by modern 3D printing hardware.
To reduce the number of degrees of freedom for simulation and optimization, \citet{rodrigues:2002:h} proposed an interesting formulation where microstructure designs 
and macroscopic layouts of the underlying microstructures are hierarchically coupled and treated simultaneously.
This initial work has been extended in multiple ways~\citep{coelho:2008:h,nakshatrala:2013:non,yan:2014:concurrent,xia:2014:reduced}.
However, these methods still need to handle variables defined at
the microstructure level and therefore they remain relatively costly.
The closest to our work is probably the method proposed by~\citet{xia:2015:multiscale},
which also relies on a database to speed up computations.
However, their work specifically targets minimum compliance problem in the structural design which allows them to approximate the macroscale behaviour of the microstructures with a
particular strain-based interpolating function.
\section{Thesis Overview}\label{ch1:desc}
