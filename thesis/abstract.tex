% $Log: abstract.tex,v $
% Revision 1.1  93/05/14  14:56:25  starflt
% Initial revision
% 
% Revision 1.1  90/05/04  10:41:01  lwvanels
% Initial revision
%
%% It will be single-spaced and the rest of the text that is supposed to go on
%% the abstract page will be generated by the abstractpage environment.  This
%% file should be \input (not \include 'd) from cover.tex.
Modern manufacturing technologies such as 3D printing enable the fabrication of objects with extraordinary complexity.
Arranging materials to form functional structures can achieve a much wider range of physical properties than in the constituent materials.
Many applications have been demonstrated in the fields of mechanics, acoustics, optics, and electromagnetics.
Unfortunately, it is difficult to design objects manually in the large combinatorial space of possible designs.
Computational design algorithms have been developed to automatically design objects with specified physical properties.
However, many types of physical properties are still very challenging to optimize because predictive and efficient simulations are not available for 
problems such as high-resolution non-linear elasticity or dynamics with friction and impact.
For simpler problems such as linear elasticity, where accurate simulation is available,
the simulation resolution handled by desktop workstations is still orders of magnitudes below available printing resolutions.

We propose to speed up simulation and inverse design process of fabricable objects by using multiscale methods.
Our method computes coarse-scale simulation meshes with data-drive material models.
It improves the simulation efficiency while preserving the characteristic deformation and motion of elastic objects.
The first step in our method is to construct a library of microstructures with their material properties such as Young's modulus and Poisson's ratio.
The range of achievable material properties is called the material property gamut.
We developed efficient sampling method to compute the gamut by focusing on finding samples near and outside the currently sampled gamut.
Next, with a pre-computed gamut, functional objects can be simulated and designed using microstructures instead of the base materials.
This allows us to simulate and optimize complex objects at a much coarser scale to improve simulation efficiency.
The speed improvement leads to designs with as many as a trillion voxels to match printer resolutions.
It also enables computational design of dynamic properties that can be faithfully reproduced in reality.
In addition to efficient design optimization, 
the gamut representation of the microstructure envelope provides a way to discover templates of microstructures with extremal physical properties.
In contrast to work where such templates are constructed by hand,
our work enables the first computational method to automatically discovery microstructure templates that arise from voxel representations.
