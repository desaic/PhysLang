% $Log: abstract.tex,v $
% Revision 1.1  93/05/14  14:56:25  starflt
% Initial revision
% 
% Revision 1.1  90/05/04  10:41:01  lwvanels
% Initial revision
% 
%
%% The text of your abstract and nothing else (other than comments) goes here.
%% It will be single-spaced and the rest of the text that is supposed to go on
%% the abstract page will be generated by the abstractpage environment.  This
%% file should be \input (not \include 'd) from cover.tex.
Modern manufacturing technologies such as 3D printing enable the design and fabrication of objects with extraordinary complexity.
By arranging materials to form functional structures, a much wider range of physical properties can be achieved compared to the constituent materials.
Many applications have been demonstrated in the fields of mechanics, acoustics, optics and electromagnetics.
Unfortunately, it is difficult to design objects manually in the large combinatorial space of possible designs.
Computational design algorithms have previously been developed to automatically design objects with specified physical properties.
However, many types of physical properties are still very challenging to optimize because predictive and efficient simulations are not available for 
problems such as high-resolution non-linear elasticity or dynamics with friction and impact.
For simpler problems such as linear elasticity where accurate simulation is available,
the simulation resolution handled by desktop workstations is still orders of magnitudes below available printing resolutions.
We propose to speed up simulation and inverse design process of fabricable objects by using multiscale methods.
First, we construct a library of microstructures and their bulk material properties.
Bulk material properties are represented by parametric material models with a small number of parameters.
To make use of the microstructure library, the space of achievable bulk parameters called the material property gamut is represented as a level set.
The level set implemented using a signed distance function supports efficient query of distance to the gamut boundary. It enables efficient sampling of the gamut by focusing on finding samples near and outside the currently sampled boundary. We use a combination of discrete and continuous sampling methods to push the structures to the limits of achievable material properties.
Next, with a pre-computed gamut, functional objects can be designed using microstructures instead of the base materials.
This allows us to simulate and optimize complex objects at a much coarser scale to improve simulation efficiency. The speed improvement leads to the designs reaching a trillion voxels to match printer resolutions. It also enables computational design of dynamic properties that can be faithfully reproduced in reality.
In addition to efficient design optimization, 
the gamut representation of the microstructure envelope provides a way to discover templates of microstructures with extremal physical properties.
In contrast to previous work where such templates are constructed by hand,
this is the first computational method to automatically discovery microstructure templates from voxel representations.
