\chapter{Conclusion}
We presented a class of multiscale methods for efficient FEM simulation of elastic materials.
Our goal is to use simulation and computational design algorithms to improve real-world designs.
To this end, we developed simulation tools for both linear and non-linear elastic materials.
Our simulation algorithms are efficient enough to be used in iterative design algorithms and also predictive enough to improve real-world designs.
The accuracy and efficiency of our methods are demonstrated by designing and fabricating functional 3D prints that meet specified static and dynamic deformation properties.

Traditionally, engineers improve the accuracy of FEM simulation by adaptively refining the mesh until some convergence test is satisfied.
The elements must be small enough to capture the geometric and material variations.
Moreover, to combat numerical stiffening, the element sizes must be a fraction of the geometric feature sizes.
Because many small elements are needed to represent a detailed design, accurate simulation can take from minutes to hours on a single computer.
Iterative design processes often requires tens or hundreds of simulations, making it impractical to use accurate simulation in the design loop.
To improve the efficiency, researchers have proposed to correct the material models instead of refining the coarse elements.
Homogenization and numerical coarsening methods are examples of this strategy where they use different rules to compute new material models for the coarse elements.
Our coarsening approach generalizes the previous methods to handle non-linear elastic materials and dynamic simulations.
The energy function of a coarse element is compactly represented as a weighted sum of basis terms.
Our simulation relies on a precomputation stage that constructs a database of material combinations and their corresponding coarse energy function parameters.
At runtime, our algorithm coarsens the high resolution mesh and then quickly finds the proper material parameters for each coarse element.
Coarsening reduces the problem size to only a fraction of the original size.
The coarse elements are simulated using precomputed material properties to maintain accuracy.
In our experiments, DDFEM achieved two-orders of magnitude speed up while still capturing the macroscopic behavior much more accurately than the baseline methods.
DAC achieved a 79x speed up of dynamic simulation while matching the trajectories of simulation and physical measurement.

To apply our simulation to computational design algorithms, we developed a set of tools for the intermediate steps.
The level set representation of the material property gamut is one such example.
It enables efficient sampling and expansion of the material property gamut.
With the gamut of 3D cubic-symmetric microstructures, we performed more analysis to identify similarities between structures with extremal properties.
This study lead to the discovery of $5$ families of auxetic microstructures.
The level set gamut has also been used in our two-scale topology optimization algorithm to constrain the material parameters at each cell.
Out algorithm can optimize much more complex models since each cell now contains material parameters that can be mapped to precomputed microstructures.
We can scale up our problem size and successfully optimize a bridge model with a trillion voxels, a resolution approaching 3D printer capabilities.
Another example is our new boundary-balancing impact (BBI) model.
It corrects the overly energetic rebound behavior of objects undergoing inelastic impact.
This makes the simulation algorithm to be predictive enough to produce the same qualitative landing behavior.

Our simulation algorithms have been used in computational design tools to generate elastic objects satisfying design goals
such as using the minimum amount of material and achieving target deformed shapes.
In addition to software validation, our simulation is extensively validated by physical experiments using fabricated objects.
For static objects, we validated their Young's modulus and Poisson's ratio using compression tests.
The fabricated compliant mechanisms also deform to target shapes.
For dynamic mechanisms, our simulation replicates the behavior of hand-tuned and optimized jumping mechanisms.
The landing poses and rebound behavior are accurately predicted.
Our experiments points the way forward for our simulations to be extended and tested for more complex deformable and dynamic systems.

\section{Limitations and Future Work}
Our work is only a first step towards efficient computational design of physical objects.
There are many potential future directions for improvements, experiments and applications.
\paragraph{More General Material Models}
Our material model still works with deformation gradients sampled at quadrature points.
However, in general, the elastic energy can be any function of the vertex positions that respects conservation laws.
One can use a much broader class of functions such as a neural network that map from a vector of vertex displacements to an energy value.
A more general function can capture the energy behavior of higher order deformations such as bending and twisting.
\paragraph{More Physical Phenomena}
Our experiments focused on elastic properties.
Engineering design problems cover a much wider range of physical phenomena:
structures with zero thermal expansion for space applications, 
efficient antenna designs from simulation of electric-magnetic field, 
tougher composite materials by mixing soft and stiff materials and so on.
Using similar coarsening techniques to speed up simulation and design of 
other physical properties remains a promising direction for exploration.
\paragraph{Incorporating Other Numerical Techniques}
Our coarsening method and other numerical methods for speed improvements are not mutually exclusive.
Since our coarsened discretization uses the same hexahedron elements as the high-resolution elements, we can combine our method with many types of techniques.
For example, our current implementation uses a GPU geometric multigrid method to greatly improve the simulation speed compared to a direct linear solver.
For future work, we can also experiment with adaptive meshing to use even coarser elements at locations where less details are required.
On the other hand, we can learn new material models for higher order elements when more details are required such as with bending and buckling.
\paragraph{High Frequency Vibration}
Our work focused on modeling the macroscopic behavior of objects such as the overall deformation and trajectories.
Our assumption is that only the low frequency vibrations have sufficient amplitude to influence the macroscopic trajectories.
High frequency vibration such as sound are unlikely to be captured accurately out of the box with our current method.
To improve the simulation accuracy of such effects, one can embed additional material information such as sound radiation models~\citep{schweickart2017animating} in the coarse elements.
\paragraph{Non-Convex Optimization}
For proof of concept, we only experimented with well-tested optimization algorithms such as topology optimization.
For dynamic problems, we applied gradient-based method to improve the design towards a local minimum.
We envision future applications such as robot design to require much more sophisticated optimization algorithms 
that samples and tweaks many types of design parameters such as control parameters, trajectory planning, geometry and material parameters etc.
These problems often have many bad local minima, requiring a systemetic sampling of different basins of attraction.
Such optimization algorithms would benefit even more from an efficient simulation due to a much larger design space to explore.
